\section{Conclusion and Future Work}
\subsection{Conclusion}

In this work, a straightforward yet highly effective CNN-based recognition system was developed for the purpose of hand vein recognition. Despite the architecture's simplicity, it demonostrated remarkable efficiency by achieving rapid training convergence and maintaining low computational requirements. The results indicate that the model achieves an accuracy of \textbf{xx.x\%} on the training set, traslating to a robust perfomance when deployed on the test set, where it obtained an accuracy of \textbf{xx.x\%}. Such a perfomance level is significant for a recognition task, as it ensures reliable identification while keeping resource consumption and development complexity to a minimum. Overall, this implementation highlights that even a relatively uncomplicated CNN framework can deliver strong accuracy and efficiency, making it an appealing solution for practical vein-based recognition systems.

\subsection{Future Work}

An interesting extension to this project for future exploration involves expanding the focus beyond the current palm-based Region of Interest (ROI). Instead of restricting the system solely to the palm vein pattern, it would be beneficial to consider additional hand characteristics such as overall hand geometry, finger length and spacing, and other distinctive biometric features of the hand. Using these complementary features would allow us to develop multiple specialized models, one dedicated to palm veins (as implemented in this project), another concentrating on hand geometry, another on finger structure, and so on. By utilizing an ensemble learning strategy (see Figure~\ref{fig:ensemble}), where each specialized model contributes its prediction, we could combine these outputs into a final, more robust decision. This multi-model, ensemble-based approach has the potential to significantly improve the overall accuracy and reliability of the system.

\begin{figure}[h]
    \centering
    \resizebox{0.4\textwidth}{!}{% Scale reduced to 40% of text width
        \begin{tikzpicture}[
            font=\sffamily,
            >=stealth,
            node distance=1cm,
            every node/.style={align=center}
        ]
            \node[circle, draw, fill=blue!20, minimum size=1cm] (sample) {New\\sample};

            \node[rectangle, draw, rounded corners, fill=gray!20, 
                  above left=1.2cm and 2cm of sample] (model1) {model 1};
            \node[rectangle, draw, rounded corners, fill=cyan!20, 
                  above=1.2cm of sample] (model2) {model 2};
            \node[rectangle, draw, rounded corners, fill=red!20, 
                  above right=1.2cm and 2cm of sample] (model3) {model 3};

            \node[above=1.0cm of model1] (pred1) {0};
            \node[above=1.0cm of model2] (pred2) {0};
            \node[above=1.0cm of model3] (pred3) {1};

            \node[above=1.4cm of pred2, font=\Large] (ens) {0};
            \node[font=\footnotesize, above=3pt of ens] (ensLabel) 
                 {Ensemble’s prediction\\(e.g., majority vote)};

            \draw [->] (sample) -- (model1);
            \draw [->] (sample) -- (model2);
            \draw [->] (sample) -- (model3);

            \draw [->] (model1) -- (pred1);
            \draw [->] (model2) -- (pred2);
            \draw [->] (model3) -- (pred3);

            \draw [->] (pred1) -- (ens);
            \draw [->] (pred2) -- (ens);
            \draw [->] (pred3) -- (ens);

        \end{tikzpicture}
    } % End of resizebox
    \caption{Illustration of an ensemble learning strategy. Each model processes the same input and produces a prediction. These predictions are combined (e.g., through majority voting) to create the ensemble's final output.}
    \label{fig:ensemble}
\end{figure}