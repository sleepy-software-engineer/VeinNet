\section{Introduction}

In recent years, hand biometrics has become an increasingly popular modality in biometric recognition systems due to its accessibility and rich discriminatory features. Traditionally, hand-based systems have relied on contact-based devices equipped with pegs or plates for image acquisition. While effective, these systems often raise hygiene concerns and reduce user acceptance. In response, there has been a shift toward contact-free systems that eliminate the need for physical contact during data capture \cite{xiong2005peg,jiang2007new,xiong2005model}. However, increased freedom of hand movement in contact-free setups often leads to reduced recognition accuracy. The integration of multispectral imaging techniques has been successfully applied to improve recognition performance in other biometric domains, such as face recognition \cite{kong2007multiscale, singh2008integrated}. Similarly, for hand biometrics, \cite{wang2006near} demonstrated that passive infrared imaging is inadequate for extracting vein patterns from the palm. This limitation has led to the exploration of active multispectral imaging across visible to near-infrared wavelengths. Previous studies, such as \cite{wang2007fusion}, have illustrated the potential of combining palmprint and palm vein images using fusion techniques applied at the image level. However, these approaches often rely on semi-touchless acquisition systems or frequency-division hardware, which may limit scalability and increase costs. 

This report proposes a method for palm vein pattern identification using a convolutional neural network (CNN) architecture. Unlike traditional approaches relying on pixel-level fusion and feature-level registration techniques, the proposed method uses the power of CNNs to automatically learn and extract distinctive features from palm vein patterns. By focusing on the rich discriminatory information in vein structures. This approach represents a significant step forward in developing efficient, hygienic, and user-friendly biometric systems.

Palm veins unique vascular patterns are highly distinctive, stable over time, and satisfy key requirements for a reliable biometric feature, such as universality, uniqueness, permanence, and measurability, making them ideal for accurate and secure biometric systems.